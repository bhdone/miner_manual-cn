\chapter{挖矿指南}
\section{初始化空间(P盘)}
\begin{flushleft}
    初始化空间,俗称P盘。请参考Chia官方的网站上与P盘相关的信息。请注意,使用Chia的钱包生成的与私钥相关的Seed,在稍后需要复制到配置文件中,用来对区块进行签名。
\end{flushleft}
\section{下载节点程序}
\begin{flushleft}
    请访问BHD的官方网站(https://bhd.one)下载最新版本的BHD钱包,安装完成后,运行钱包程序则可以开始同步区块数据。
\end{flushleft}
\section{安装}
\begin{flushleft}
    双击下载好的安装包来安装钱包程序,当钱包程序安装完成后,将可以在开始菜单中找到与BitcoinHD相关的文件夹和程序图标。并且,在你打开了命令提示符的程序后,可以直接输入btchd-miner等相关的命令来执行一些操作。
\end{flushleft}
\section{编写配置文件}
\begin{flushleft}
    在开始挖矿工作前,需要先编写一个简单的配置文件,用于指定一些与挖矿工作相关的基本信息。
\end{flushleft}
\subsection{初始化配置文件}
\begin{flushleft}
    我们使用btchd-miner.exe来进行挖矿相关的工作,初始化一份空的配置文件,是它的其中一项功能。你也可以不使用它来初始化该文件,直接把已经撰写好的配置文件复制到你平常挖矿开始的目录中即可,我们在稍后启动挖矿程序时会需要指定该文件的位置。
\end{flushleft}
\begin{flushleft}
    使用下方的命令来初始化一份空的配置文件。
\end{flushleft}
\begin{minted}{bash}
    btchd-miner.exe generate-config
\end{minted}
\subsection{填写配置}
\begin{flushleft}
    配置文件格式是Json文本,默认命名为``config.json'',下方是一个示例,你的配置文件看起来也会和下面的差不多。
\end{flushleft}
\scriptsize
\begin{minted}{json}
{
    "reward": "38CLnjuj31ifZMXZV8UhbyCo3fNP46Lszy",
    "seed": "bird convince trend skin lumber escape crater describe ...",
    "plotPath": [
        "/home/matthew/data/plotfiles1",
        "/home/matthew/data/plotfiles2"
    ],
    "rpc": {
        "host": "http://127.0.0.1:18732"
    }
}
\end{minted}
\normalsize
\subsubsection{BHD奖励地址}
\begin{flushleft}
    这是你在BHD网络中的接收挖矿奖励的地址。注意,在你使用这个地址前,需要确保已经将其与你的Chia的P盘文件相关的Farmer公钥相绑定,要了解如何进行绑定操作,请查阅``绑定FarmerId''一节。
\end{flushleft}
\subsubsection{Chia钱包seed}
\begin{flushleft}
    你在初始化硬盘空间的时候,Chia会需要你提供一个私钥,这个私钥由这个叫做``Seed''的字符串组成,里面是一些单词的组合。我们在产生BHD的区块时,需要用到这个私钥来对区块进行签名,用以验证你的身份。请将该``Seed''粘贴到这时。该私钥只用于签名,并且只保存在本地,不会被上传到BHD的链上或者网络中。
\end{flushleft}
\subsubsection{Plot文件目录}
\begin{flushleft}
    在这里你需要指定你的Plot文件目录,该登记项是一个数组,你可以同时登记多个Plot文件目录。
\end{flushleft}
\subsubsection{本地钱包的RPC连接}
\begin{flushleft}
    与本地钱包的RPC连接,如果你使用登录用户名称和密码,可以在这里指定。默认的我们使用``.cookie''文件来完成与钱包的通讯认证,所以在这里只需要登记``host''项即可。
\end{flushleft}
\section{绑定FarmerId}
\begin{flushleft}
    在挖矿开始前,你需要把你的BHD接收奖励的地址和FarmerId绑定,因为需要做签名,所以你需要先编写正确的配置文件并将你正确的``Seed''和BHD奖励地址填入。使用下方的命令来绑定FarmerId。
\end{flushleft}
\scriptsize
\begin{minted}{bash}
    btchd-miner.exe bind --test
\end{minted}
\normalsize
\begin{itemize}
    \item ``bind''为主命令
    \item ``--test''参数,使用测试网络,若没有该参数则命令在主网上执行
\end{itemize}
\section{质押BHD}
\begin{flushleft}
    将BHD质押到网络上,以获取区块的完整收益,使用下面的命令来进行质押。
\end{flushleft}
\scriptsize
\begin{minted}{bash}
    btchd-miner.exe deposit --amount [质押数量] --term [noterm, term1, term2, term3] --testnet
\end{minted}
\normalsize
\begin{itemize}
    \item 其中``deposit''为主命令
    \item ``--amount''参数,用于指定总共质押的货币数量
    \item ``--term''参数,用于指定锁定的期限。其中:``noterm''为活期, ``term1'', ``term2'', ``term3''分别为不同的锁定期限类型
    \item ``--testnet''参数,使用测试网络,若没有该参数则命令在主网上执行
\end{itemize}
\begin{flushleft}
    在质押成功后,btchd-miner.exe会显示一个名为``交易id''的哈希值,请将其记下来,在未来需要将质押金额取出的时候,要用到这个``交易id''。
\end{flushleft}
\section{退出质押}
\begin{flushleft}
    当质押的金额的锁定期限到达后,矿工们可以将质押的金额取出。使用下方的命令来取出。
\end{flushleft}
\scriptsize
\begin{minted}{bash}
    btchd-miner.exe withdraw --txid [交易id] --testnet
\end{minted}
\normalsize
\begin{itemize}
    \item 其中``withdraw''为主命令
    \item ``--txid''参数,指定``交易id'',这个``交易id''在之前使用``deposit''命令成功后会被显示出来。
\end{itemize}
\section{启动挖矿}
\subsection{启动节点程序}
\begin{flushleft}
    节点程序是整个BitcoinHD的核心,它用于区块的下载,校验以及产生。它同时也管理着本地钱包、私钥,对区块进行签名,维护P2P网络等重要的工作。若要处理网络事务,包含区块下载、校验和挖矿等,该节点程序需要一直保持运行状态。要启动节点程序,请使用下方的命令。
\end{flushleft}
\scriptsize
\begin{minted}{bash}
    btchdd.exe -testnet -server
\end{minted}
\normalsize
\begin{itemize}
    \item ``-testnet''参数,表示启动测试网络的节点程序,若没有该参数则启动正式网络的节点程序
    \item ``-server''参数,打开RPC服务器,用于稍后与挖矿程序进行通信
\end{itemize}
\subsection{启动挖矿程序}
\begin{flushleft}
    当一切都准备就绪后,包括节点已经将所有的区块都同步完成,则可以启动挖矿程序。使用下方的命令来启动。
\end{flushleft}
\scriptsize
\begin{minted}{bash}
    btchd-miner.exe mining --testnet
\end{minted}
\normalsize
\begin{itemize}
    \item ``mining''为主命令
    \item ``--testnet''参数,使用测试网络,若没有该参数,则命令在主网上执行
\end{itemize}
