\chapter{挖矿指南}
\section{初始化空间(P盘)}
\begin{flushleft}
    初始化空间,俗称P盘。请参考Chia官方的网站上与P盘相关的信息。请注意,使用Chia的钱包生成的与私钥相关的Seed,在稍后需要
    复制到配置文件中,用来对挖到的区块进行数字签名。
\end{flushleft}
\section{下载节点程序}
\subsection{Windows}
\begin{flushleft}
    若你使用的操作系统是Windows,请访问BHD的官方网站(https://bhd.one)下载最新版本的BHD节点钱包安装文件并双击进行安装。
    安装完成后,运行钱包程序则可以开始同步区块数据。注意,如果你在使用测试网络,请点击``绿色''的应用程序图标来运行带图形
    操作界面的对应测试网络的节点钱包程序。带黄色图标的程序是对应正式网络的节点钱包,在测试阶段请不要运行该程序。
\end{flushleft}
\begin{flushleft}
    你也可以运行控制台版本的节点钱包程序。在命令行执行``\mintinline{bash}{btchdd -testnet -server}''来启动控制台节点钱包。
    参数``\mintinline{bash}{-testnet}''表明启动该节点并连接至测试网络。
\end{flushleft}
\subsection{Linux}
\begin{flushleft}
    若你使用的操作系统是Linux,请在官方网站下载对应的安装包。使用tar命令将下载包里的文件解出,然后将其中所有的文件拷贝至
    ``/usr/local/bin''目录中。Linux压缩包中同时包含控制台版本和图形界面版本的节点钱包,运行方式和Windows下一致。
\end{flushleft}
\subsection{macOS}
\begin{flushleft}
    macOS的安装包是扩展名为.dmg的文件,在macOS中双击该文件并且将里面的BitcoinHD图标拖动到Application文件夹中完成安装。剩下
    的工作和Windows/Linux一致,你需要找到带绿色图标的节点钱包并运行,或者使用控制台版本的节点钱包,同Windows。
\end{flushleft}
\section{安装 - Windows篇}
\begin{flushleft}
    双击下载好的安装包来安装钱包程序,当钱包程序安装完成后,将可以在开始菜单中找到与BitcoinHD相关的文件夹和程序图标。并且,
    在你打开了命令提示符的程序后,可以直接输入btchd-miner等相关的命令来执行操作。
\end{flushleft}
\subsection{添加搜索路径}
\begin{flushleft}
    安装完成后,你需要将BitcoinHD的daemon文件夹添加到系统的路径搜索列表里,具体操作如下:
\end{flushleft}
\begin{enumerate}
    \item 将BitcoinHD的安装目录的子目录``daemon''在资源管理器中复制下来,通常为
        ``C:\symbol{92}Program Files\symbol{92}BitcoinHD\symbol{92}daemon'';
    \item 依次打开:Windows开始菜单-设置-系统-关于-高级系统设置-环境变量;
    \item 在打开的窗口里有两个列表,在上方的第一个列表中找到``path''项目,双击它打开路径编辑窗口;
    \item 点击``新建''按钮,将复制好的目录粘贴到列表中,然后点击确定保存;
    \item 重新打开你的命令行窗口,这时输入``\mintinline{bash}{btchd-miner --help}''后应该可以看到命令被正确调用;
\end{enumerate}
\subsection{配置节点RPC服务}
\begin{flushleft}
    默认的情况下,如果你使用GUI版本的钱包,它是不能直接与挖矿程序进行通讯的。如果你需要使用GUI版本的钱包来挖矿,则需要修
    改其配置:
\end{flushleft}
\begin{enumerate}
    \item 打开钱包,注:若是使用测试网络,请双击绿色的那个图标;
    \item 依次打开:设置-选项-配置文件,并且在弹出的提示框中点击``确定'';
    \item 在打开的文本编辑器中输入``server=1'',然后保存;
    \item 关闭钱包并重启,此时挖矿程序将可以正确连接;
\end{enumerate}
\section{安装 - Linux篇}
\subsection{下载与安装}
\begin{flushleft}
    首先使用wget或curl将Linux版本的程序包下载到本地,然后使用``tar xf 程序包路径''将包中所有的文件解开。然后再将这些命令
    复制到``/usr/local/bin''目录中。注意:你若不是root用户,那么需要使用``sudo''命令来执行拷贝工作。如下所示:
\end{flushleft}
\scriptsize
\begin{minted}{bash}
$ sudo cp btchd/* /usr/local/bin/
\end{minted}
\normalsize
\begin{flushleft}
    此时,输入``\mintinline{bash}{btchd-miner --help}''会显示出挖矿程序的基本使用说明。Linux版本的钱包节点程序通过命令行
    参数则可以进行配置,不需要额外修改配置文件。但是挖矿的配置程序是必需的,后续的章节若没有做特殊说明,则表示这些配置和
    操作在各操作系统下是相同的。
\end{flushleft}
\subsection{配置chiavdf运行时间证明}
\subsubsection{什么是时间证明?}
\begin{flushleft}
    BHD这次升级不仅使用了chia的PoS容量证明,同时也使用了PoT时间证明。通过时间证明的结果将更好的保证网络的安全性。
    每次矿工产生新的区块,都将被要求在该区块中包含一个或多个时间证明的结果。这个时间证明的结果可以通过运行时间证明计算器
    来获得。
\end{flushleft}
\subsubsection{我要运行时间证明计算器吗?}
\begin{flushleft}
    时间证明运算器是一个可选的组件,但是如果在本地运行时间证明运算器,那么则可以最快的获取结果,更有利于当前的出块操作。
    若不运行时间证明,那么所需的证明结果会通过P2P网络去请求其它运行了时间证明的节点来获得,这样可能会因为网络延迟等其它的
    原因导致出块延迟或出不了块。
\end{flushleft}
\subsubsection{时间证明运算器运行条件}
\begin{itemize}
    \item 操作系统需要Linux或macOS;
    \item 机器中有一块较好的cpu,若cpu主频太差,那么时间证明的结果获取太慢,也没有意义;
\end{itemize}
\subsubsection{测试电脑的vdf计算速度}
\begin{flushleft}
    若你自行编译了工程chiavdf,那么可以在chiavdf/src目录下找到一个名为vdf\_bench的可执行文件,运行该文件可以获得当前电脑
    vdf运算的速度,使用命令``\mintinline{bash}{vdf_bench square_asm 250000}'',该命令会让vdf\_bench执行指定数量的计算,
    然后获得每秒的vdf计算量并显示出来。例如,我在一台MacbookPro A1707上运行得到的结果是每秒150k ips左右,而在一台将近10年
    的老式PC上计算的结果是130k ips左右,而在一台安装了i5 10600kf的电脑上得到的结果是每秒200k ips左右。
\end{flushleft}
\subsubsection{下载时间证明运算器}
\begin{flushleft}
    BHD直接延用chia的时间证明运算器,你可以自行下载该项目的源代码并在Linux本地进行编译。或者在BHD的官网上下载我们为你编译
    好的版本。该项目的源代码的github地址为:``https://github.com/chia-network/chiavdf''。该项目将会通过编译生成一个名为
    ``vdf\_client''的文件,该文件就是时间证明运算器,是一个可执行程序,后续启动节点时需要指定该文件的完整路径。
\end{flushleft}
\subsection{启动节点并运行时间证明运算器}
\begin{flushleft}
    若不打算在本地运行时间证明运算器,则在启动钱包节点时只需要带上``\mintinline{bash}{btchdd -server}''即可(若是测试网络
    则需要加上``\mintinline{bash}{-testnet}'')。若打算运行时间证明运算器,那么请同时添加
    ``\mintinline{bash}{-timelord -timelord-vdf_client}=[the full path to vdf\_client binary]''。
\end{flushleft}
\begin{itemize}
    \item \mintinline{bash}{-timelord},该参数表示启动时间证明运算器;
    \item \mintinline{bash}{-timelord-vdf_client},该参数指明运算器的完整路径;
\end{itemize}
\subsection{使用Timelord}
\begin{flushleft}
    Timelord是一个桥梁程序,用于执行vdf\_client并获得计算结果并返回给Miner,它与节点钱包启动时通过过自带timelord参数来
    获取计算结果相比,有以下几个优点:
\end{flushleft}
\begin{itemize}
    \item 不依附节点钱包,直接通过socket与Miner连接,获得的结果直接返回给Miner出块打包;
    \item 结果不再上传至P2P网络,不再造成P2P网络中数据包过多的问题;
    \item 其被编译打包成为Docker容器,使其可以在Windows下被直接运行,Windows的矿工也可以通过其获得vdf计算结果;
    \item BHD社区也会推出公共的timelord服务,供那些不运行timelord的用户获取vdf证明结果;
\end{itemize}
\subsubsection{启动Timelord}
\begin{flushleft}
    Timelord可以在Linux下被编译运行,若你使用Windows操作系统,请参考``本地使用Docker来运行''那一章节的内容。
\end{flushleft}
\begin{flushleft}
    请前往(https://bhd.one)相关页面下载已经编译好的Linux包,并且将其中的文件解压缩至可运行目录,通常是``/usr/local/bin''。
    解压出来有三个可执行的文件,分别为:timelord, vdf\_client, vdf\_bench。其中timelord是主执行文件,用于执行timelord服务,
    vdf\_client由timelord调用获取vdf答案,vdf\_bench用来测试当前的电脑获取vdf答案的速度。
\end{flushleft}
\begin{flushleft}
    timelord的工作流程如下:首先,timelord通过socket连接至节点钱包的RPC服务,通过节点钱包查询到当前网络中最新的区块的
    challenge。同一时间,miner可以连接至timelord并且向timelord请求计算。timelord会检查miner的请求中的challenge,若和当前
    网络中的challenge一至且相关的计算没有开始,则会开始启动计算。多个miner请求的iters的值不同,所以当对应某个iters的答案
    计算出来后,将直接通过socket返回给对应的miner。若节点钱包的最新区块已经改变,则需要停止当前的计算,准备启动下一个计算
    的服务。
\end{flushleft}
\begin{flushleft}
    使用``\mintinline{bash}{--help}''可以获得timelord命令的帮助信息,timelord的命令主要分为:节点钱包连接参数,miner服务
    参数和vdf\_client相关参数。
\end{flushleft}
\scriptsize
\begin{minted}{bash}
Timelord, run vdf and returns proofs
Usage:
  Timelord [OPTION...]

      --help                 Show help document
      --logfile arg          Store logs into file (default: ./timelord.log)
  -v, --verbose              Show more logs
      --addr arg             Listening to address (default: 127.0.0.1)
      --port arg             Listening on this port (default: 19191)
      --vdf_client-path arg  The full path to `vdf_client` (default:
                             $HOME/chiavdf/src/vdf_client)
      --vdf_client-addr arg  vdf_client will listen to this address
                             (default: 127.0.0.1)
      --vdf_client-port arg  vdf_client will listen to this port (default:
                             29292)
      --rpc arg              The endpoint of btchd core (default:
                             http://127.0.0.1:18732)
      --cookie arg           Full path to the `.cookie` file generated by
                             btchd core (default:
                             $HOME/.btchd/testnet3/.cookie)
      --rpc-user arg         The username to identify RPC connection
      --rpc-password arg     The password to verify RPC connection
\end{minted}
\normalsize
\begin{flushleft}
    需要关心的参数首先是参数``\mintinline{bash}{--vdf_client-path}'',该参数指定你的vdf\_client所在的完整路径。其次,你
    需要指定节点钱包的RPC连接地址,通过参数``\mintinline{bash}{--rpc}''来指定连接的目的地,注意,若你的节点程序没有设置
    用户名和密码,那么你需要指定``\mintinline{bash}{.cookie}''所在的位置,否则请通过``\mintinline{bash}{--rpc-user}''和
    ``\mintinline{bash}{--rpc-password}''来指定用户名和密码。
\end{flushleft}
\subsubsection{Miner如何连接Timelord}
\begin{flushleft}
    Miner中有三个参数用于指定其需要连接的timelord服务,分别为:``\mintinline{bash}{--timelord}'',
    ``\mintinline{bash}{--timelord-host}'',``\mintinline{bash}{--timelord-port}'',第一个参数表示需要与timelord连接来获取
    vdf答案,后面两个参数分别指定timelord的连接地址及端口。
\end{flushleft}
\subsubsection{本地使用Docker来运行}
\begin{flushleft}
    首先下载并安装Docker,相关的信息可以在(https://docker.io)找到。安装好Docker之后,在命令行窗口可以输入docker来执行相关
    的命令。这时,可以执行命令:``\mintinline{bash}{docker pull bhdone/timelord}''来下载已经编译好的docker包到本地。为了
    方便起见,该镜像包同时包含有timelord, vdf\_client和vdf\_bench。
\end{flushleft}
\begin{flushleft}
    首先,需要配置节点钱包,使其RPC服务需要使用用户名和密码来登录,添加参数:``\mintinline{bash}{-rpcuser=hello}'',和
    ``\mintinline{bash}{-rpcpassword=world}'',这样就设置了RPC服务的登录用户名和密码。
\end{flushleft}
\begin{flushleft}
    然后,在命令行窗口中输入下方命令启动Docker镜像中的timelord,注意要将与用户名和密码相关的参数修改成你之前设置的值。以下命令会在本地打开19191
    端口,以供本地的miner程序连接,如果需要侦听0.0.0.0请修改对应的参数即可。关于Docker的参数使用说明可以在(https://docker.io)上自行查阅,
    这里就不再赘述。
\end{flushleft}
\scriptsize
\begin{minted}{bash}
$ docker run --add-host host.docker.internal:host-gateway -p 127.0.0.1:19191:19191
    -it bhdone/timelord /timelord --rpc=http://host.docker.internal:18732
    --rpc-user=hello --rpc-password=world
    --vdf_client-path=/vdf_client
    --bind=0.0.0.0 --port=19191
\end{minted}
\normalsize
\subsubsection{连接至Docker版本的Timelord}
\begin{flushleft}
    按照上方的Timelord启动命令,Timelord已经侦听了19191端口,那么使用Miner连接至Timelord可以使用命令\scriptsize
    ``\mintinline{bash}{$ miner --timelord --timelord-host=127.0.0.1 --timeload-port=19191}''\normalsize
\end{flushleft}
\subsubsection{连接至公共的Timelord}
\begin{flushleft}
    连接至公共的Timelord,可以直接在miner的参数添加公共的timelord的地址和端口即可,假设有一台timelord服务器的hostname为:
    timelord.bhd.one,端口为:19191,那么可以使用下方的命令来进行连接。
\end{flushleft}
\scriptsize
\begin{minted}{bash}
$ btchd-miner mining --timelord --timelord-host=timelord.bhd.one --timelord-port=19191
\end{minted}
\normalsize
\subsubsection{速度测试}
\begin{flushleft}
    通过Docker命令来运行vdf\_bench也可以进行速度测试。
\end{flushleft}
\scriptsize
\begin{minted}{bash}
$ docker run -it timelord /vdf_bench square_asm 250000
\end{minted}
\normalsize
\begin{flushleft}
    速度测试命令执行完成后同样也会显示出当前电脑每秒钟运行的vdf计算数量。但是,值得注意的是,同样一台电脑使用Docker来运行
    vdf计算会有一定的速度损失(20\%左右),配置一台Linux或macOS来运行vdf计算是一个比较好的选择。
\end{flushleft}
\subsection{一个完整的节点启动示例}
\scriptsize
\begin{minted}{bash}
$ btchdd -server -testnet -timelord -timelord-vdf_client=/home/matthew/chiavdf/src/vdf_client
\end{minted}
\normalsize
\section{编写挖矿配置文件}
\begin{flushleft}
    在开始挖矿工作前,需要先编写一个简单的配置文件,用于指定一些与挖矿工作相关的基本信息。
\end{flushleft}
\subsection{初始化配置文件}
\begin{flushleft}
    我们使用btchd-miner来进行挖矿相关的工作,初始化一份空的配置文件,是它的其中一项功能。你也可以不使用它来初始化该文件,
    直接把已经撰写好的配置文件复制到你平常挖矿开始的目录中即可,我们在稍后启动挖矿程序时会需要指定该文件的位置。
\end{flushleft}
\begin{flushleft}
    使用下方的命令来初始化一份空的配置文件。
\end{flushleft}
\scriptsize
\begin{minted}{bash}
$ btchd-miner generate-config
\end{minted}
\normalsize
\begin{flushleft}
    使用任何的编辑器都可以编辑这个文件。关于如何修改这个文件,请查阅下一小节的内容。
\end{flushleft}
\subsection{填写配置}
\begin{flushleft}
    配置文件格式是Json文本,默认命名为``config.json'',下方是一个示例,你的配置文件看起来也会和下面的差不多。
\end{flushleft}
\scriptsize
\begin{minted}{json}
{
    "testnet": true,
    "noproxy": true,
    "reward": "38CLnjuj31ifZMXZV8UhbyCo3fNP46Lszy",
    "seed": "bird convince trend skin lumber escape crater describe ...",
    "plotPath": [
        "/home/matthew/data/plotfiles1",
        "/home/matthew/data/plotfiles2"
    ],
    "rpc": {
        "host": "http://127.0.0.1:18732"
    }
}
\end{minted}
\normalsize
\subsubsection{测试网络``testnet''}
\begin{flushleft}
    当其值为true时,使用测试网络,否则使用正式网络。
\end{flushleft}
\subsubsection{不使用网络代理``noproxy''}
\begin{flushleft}
    该值为true时,表示不使用网络代理来连接钱包,若无特殊的代理需求,请保持该选项值为true。
\end{flushleft}
\subsubsection{BHD奖励地址``reward''}
\begin{flushleft}
    这是你在BHD网络中的接收挖矿奖励的地址。注意,在你使用这个地址前,需要确保已经将其与你的Chia的P盘文件相关的Farmer公钥
    相绑定,要了解如何进行绑定操作,请查阅``绑定FarmerId''一节。
\end{flushleft}
\subsubsection{Chia钱包助词词``seed''}
\begin{flushleft}
    你在初始化硬盘空间的时候,Chia会需要你提供一个私钥,这个私钥由这个叫做``Seed''的字符串组成,里面是一些单词的组合。
    我们在产生BHD的区块时,需要用到这个私钥来对区块进行签名,用以验证你的身份。请将该``Seed''粘贴到这时。该私钥只用于签名,
    并且只保存在本地,不会被上传到BHD的链上或者网络中。
\end{flushleft}
\subsubsection{Plot文件目录``plotPath''}
\begin{flushleft}
    在这里你需要指定你的Plot文件目录,该登记项是一个数组,你可以同时登记多个Plot文件目录。
\end{flushleft}
\subsubsection{本地钱包的RPC连接``rpc''}
\begin{flushleft}
    与本地钱包的RPC连接,如果你使用登录用户名称和密码,可以在这里添加``user''和``password''选项。默认的我们使用
    ``.cookie''文件来完成与钱包的通讯认证,所以在这里只需要登记``host''项即可,``user''和``password''两个选项在使用cookie
    登录方式的情况下不需要指定。
\end{flushleft}
\section{基本参数介绍}
\subsection{节点程序``btchdd''}
\subsubsection{参数}
\begin{itemize}
    \item ``-testnet''参数,启动测试网络的节点程序,若不带该参数则表示启动正式网络的节点程序
    \item ``-server''参数,启动RPC服务器,用以接收其它客户端,包括挖矿客户端的连接和命令
    \item ``-timelord''参数,启动时间证明
    \item ``-timelord-vdf\_client''参数,当指明了``-timelord''参数时,需要同时添加本参数来指定``vdf\_client''的路径
\end{itemize}
\begin{flushleft}
    注:与timelord相关的参数只能在Linux下运行,同时你需要编译chiavdf项目。``timelord''相关的参数,用于在本地计算并获取
    时间证明。关于时间证明的相关信息,可以参考BHD新版本的白皮书中与VDF相关的章节。另外,因为VDF计算暂时还没有Windows版本,
    若要运行VDF则需要使用Linux操作系统。因为每次出块,都需要获取与当前PoS相关的一个时间证明,所以,若在本地运行VDF则可以
    确保自己以在第一时间收到正确的证明答案。若没有在本地运行时间证明,那么需要通过网络将证明请求发送至可以计算证明的节点,
    在节点获得答案后会将其发回到本地。我们鼓励节点在本地运行时间证明,这样对于矿工本身和整个网络来说都是有益的。
\end{flushleft}
\subsection{挖矿程序``btchd-miner''}
\begin{flushleft}
    挖矿程序的参数由``主命令''和``命令参数''两部分组成。``主命令''指明当前要做的工作类型,``命令参数''将补全当前操作相关的
    参数。
\end{flushleft}
\subsubsection{命令列表}
\begin{flushleft}
    在执行挖矿程序时,使用参数``\mintinline{bash}{--help}''将显示出帮助文档。下方是一些重要的命令:
\end{flushleft}
\begin{itemize}
    \item ``mining''开始挖矿
    \item ``bind''绑定FarmerId
    \item ``deposit''质押金额
    \item ``withdraw''取出之前质押的金额
    \item ``retarget''将绑定重新指向另一个账号
\end{itemize}
\subsection{钱包密码}
\begin{flushleft}
    若你的钱包设置了密码,那么在使用矿工工具前需要对钱包进行解锁,否则矿工工具在使用钱包对交易记录进行签名时会出现错误。
    我们可以使用``btchd-cli''命令进行解锁操作,参考以下示例:
\end{flushleft}
\scriptsize
\begin{minted}{bash}
$ btchd-cli walletpassphrase "wallet password" 100000000
\end{minted}
\normalsize
\begin{flushleft}
    请将``wallet password''改成你的钱包密码,后面那个数字是解锁状态保持多少秒,
    设置一个较长的数值可以让你的钱包在一段时间内都处于可用的状态。
\end{flushleft}
\section{绑定FarmerId}
\begin{flushleft}
    在挖矿开始前,你需要把你的BHD接收奖励的地址和FarmerId绑定,因为需要做签名,所以你需要先编写正确的配置文件并将你正确的
    ``Seed''和BHD奖励地址填入,然后使用下方的命令来进行绑定。
\end{flushleft}
\scriptsize
\begin{minted}{bash}
$ btchd-miner bind
\end{minted}
\normalsize
\begin{itemize}
    \item ``bind''为主命令
\end{itemize}
\section{查询绑定}
\begin{flushleft}
    当绑定完成后,可以通过``\mintinline{bash}{bind --check}''来查询绑定记录。
\end{flushleft}
\scriptsize
\begin{minted}{bash}
$ btchd-miner bind --check
  --> txid: b6037724221c1a21b183596f25cfcf46da90ce87c67f4fb93d74dce5dd977463
    height: 200003
   address: 2NGWAccrksGM4TmefLN4qyW1kV7VpMngtBQ
    farmer: a7ecb9581e69e4ce968e5465764f29f519901d9bc892da89e3048b87ba820c8b04e17d726bfbb236e3f0e33f8a83851e
     valid: yes
    active: yes
\end{minted}
\normalsize
\begin{flushleft}
    查询出来的绑定交易信息包括:交易哈希``txid'',高度``height'',绑定出块地址``address''及chia的``farmer''公钥。另外还包
    括有效性标志``valid'',是否激活``active''。
\end{flushleft}
\section{质押BHD}
\begin{flushleft}
    将BHD质押到网络上,以获取区块的完整收益,使用下面的命令来进行质押。
\end{flushleft}
\scriptsize
\begin{minted}{bash}
$ btchd-miner deposit --amount 质押数量 --term [noterm, term1, term2, term3]
\end{minted}
\normalsize
\begin{itemize}
    \item 其中``deposit''为主命令
    \item ``\mintinline{bash}{--amount}''参数,用于指定总共质押的货币数量
    \item ``\mintinline{bash}{--term}''参数,用于指定锁定的期限。其中:``noterm''为活期, ``term1'', ``term2'', ``term3''
        分别为不同的锁定期限类型
\end{itemize}
\begin{flushleft}
    在质押成功后,``btchd-miner''会显示出一个哈希值,这个哈希值是本次交易的``交易哈希'',请保存下来,未来需要退出质押或者
    重新绑定质押地址的时候使用。
\end{flushleft}
\section{质押转移}
\begin{flushleft}
    质押在链上的BHD将无法在约定的时间内被全额取出,但是可以将其转移给其它的矿工地址。
\end{flushleft}
\scriptsize
\begin{minted}{bash}
$ btchd-miner retarget --txid 交易哈希 --address 地址
\end{minted}
\normalsize
\begin{itemize}
    \item 其中``retarget''为主命令
    \item ``\mintinline{bash}{--txid}''参数,用于指定将要被重新绑定的质押交易哈希
    \item ``\mintinline{bash}{--address}''参数,用于指定新的被绑定的地址
\end{itemize}
\begin{flushleft}
    为了防止在短时间内某笔质押被频繁重绑,绑定操作有时间间隔规定。该次绑定与质押或上次绑定之间的区块数量不得小于
    一个既定值。在testnet3测试网上,该数值为10,在正式网络上为7个星期左右。一笔已经重新绑定过的质押可以被再次重绑,
    命令格式与上方描述相同,需要指定上一次重绑的交易哈希。另外,只有该质押的拥有者才可以转移该质押。
\end{flushleft}
\section{查询质押}
\begin{flushleft}
    你可以通过使用``\mintinline{bash}{deposit --check}''命令来查询与当前账号相关的质押,通过这个命令你可以了解到相关的
    txid并将其用在质押命令上。在命令行中显示的内容是一个json的文本,如下方的示例显示,可以看到质押的交易哈希,金额,区块
    高度,类型等信息。
\end{flushleft}
\scriptsize
\begin{minted}{bash}
$ btchd-miner deposit --check --valid
200381 [   point  ] 0b2292e644a34b0793dd5ad1ef34ea2f7fc9fa3e19823b5d810bc8c8d46e7373
--> 2NGWAccrksGM4TmefLN4qyW1kV7VpMngtBQ      12,345 BHD [  term1 ]
2,469 BHD (actual)        111 BHD (withdraw)
  --   [   point  ] e2e2aa173b5502b63d1c886468b5dd2d96c4018a21b7db7ea26f7169f635320e
--> 2NGWAccrksGM4TmefLN4qyW1kV7VpMngtBQ      32,323 BHD [  term3 ]
2,585 BHD (actual)     32,323 BHD (withdraw)
\end{minted}
\normalsize
\begin{flushleft}
    以上是一台测试机上关于质押的查询结果,一共查询出两条质押记录。其中,第一条记录被打包到链上,区块高度为200381,
    已经起效,第二条记录因为已经被取出,所以已经失效。以下为各栏目的说明:
\end{flushleft}
\begin{itemize}
    \item 区块高度,这条质押交易被打包在哪个区块里,若区块高度显示为\mintinline{bash}{---}则说明该交易没有被打包到区块中,
        或其已经失效;注:质押交易需要被打包到区块中才能起效;
    \item 质押类型,point表示为未被转移的质押,retarget表示质押被转移过一次或多次;
    \item 交易哈希,这次交易的哈希值,全网唯一。可以复制该哈希值用于质押转移;
    \item 矿工地址,享有该质押的出块地址。注意,该地址与质押者可以是不同的地址,在质押时可以指定该质押地址;
    \item 质押金额及类型,这笔质押总共的质押量及质押的锁定类型,分别为:noterm-无,term1-锁定类型1,term2-锁定类型2,
        term3-锁定类型3。具体的锁定类型对应的期限可以查询白皮书或相关文档,正式网络和测试网络期限不同;
    \item actual - 根据锁定的类型,该笔质押在当前的网络上实际的质押金额;
    \item withdraw - 若现在取出该笔质押,实际获得的BHD数额,其它的数额将会在链上被销毁;
\end{itemize}
\section{退出质押}
\begin{flushleft}
    当质押的金额的锁定期限到达后,矿工们可以将质押的金额取出。使用下方的命令。
\end{flushleft}
\scriptsize
\begin{minted}{bash}
$ btchd-miner withdraw --txid 交易哈希
\end{minted}
\normalsize
\begin{itemize}
    \item 其中``withdraw''为主命令
    \item ``\mintinline{bash}{--txid}''参数,指定``交易哈希'',这个``交易哈希''在之前使用``deposit''命令成功后会被显示出来。
\end{itemize}
\subsection{提早退出}
\begin{flushleft}
    提早退出质押是被允许的,但是会按照质押不足相应销毁掉一定数量的BHD。销毁的BHD将退出流通,总的BHD发行量将会减少,
    并且同时影响矿工所需要的质押货币量。例如:假设你质押了一笔为期3年的$1000_{BHD}$,但是在质押时间刚到一年的时候,
    你将这笔质押的金额取出,该质押将只有$333_{BHD}$被取出,而其余的$667_{BHD}$将在链上被直接销毁。
\end{flushleft}
\section{启动挖矿}
\subsection{启动节点程序}
\begin{flushleft}
    节点程序是整个BitcoinHD的核心,它用于区块的下载,校验以及产生。它同时也管理着本地钱包、私钥,对区块进行签名,维护P2P
    网络等重要的工作。若要处理网络事务,包含区块下载、校验和挖矿等,该节点程序需要一直保持运行状态。要启动节点程序,请
    使用下方的命令。
\end{flushleft}
\scriptsize
\begin{minted}{bash}
$ btchdd -testnet -server
\end{minted}
\normalsize
\begin{itemize}
    \item ``\mintinline{bash}{-testnet}''参数,表示启动测试网络的节点程序,若没有该参数则启动正式网络的节点程序
    \item ``\mintinline{bash}{-server}''参数,打开RPC服务器,用于稍后与挖矿程序进行通信
\end{itemize}
\subsection{启动挖矿程序}
\begin{flushleft}
    当一切都准备就绪后,包括节点已经将所有的区块都同步完成,则可以启动挖矿程序。使用下方的命令来启动。
\end{flushleft}
\scriptsize
\begin{minted}{bash}
$ btchd-miner mining
\end{minted}
\normalsize
\begin{itemize}
    \item ``mining''为主命令
\end{itemize}
